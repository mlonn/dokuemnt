\section{Allmänt}

\subsection{Syfte}
Ölbrukets syfte är att verka för spridande av god ölkultur och umgängeskonst bland Chalmerister och Ölbruksvänner. Ölbruket ämnar utveckla teknologens kunnande och intresse för öl och därigenom förstå skillnader inom det stora utbud av ölsorter som finns. Detta uppnås genom att arrangera aktiviteter för föreningens medlemmar som föreläsningar, provningar, bryggeribesök eller liknande. Föreningen bör i denna anda även skapa förutsättningar för lämpliga öldistributions-, öllagrings och ölkonsumtionssystem ämnade att befrämja god kultur, konst och smak.

\subsection{Oberoende}
Ölbruket är en partipolitiskt obunden samt religiöst neutral förening.

\subsection{Säte}
Ölbruket har sitt säte i Göteborg.

\subsection{Ekonomi}
Föreningens firma tecknas enskilt av styrelsens ordförande och kassör. Kassören har hand om föreningens ekonomi.

\subsection{Verksamhets- och räkenskapsår}
Ölbruket har verksamhets- och räkenskapsår 1 juli - 30 juni.

\subsection{Stadgeändringar}
Två efter varandra medlemsmöten krävs för att ändra stadgarna och minst två tredjedelars majoritet krävs på båda dessa möten.

\subsection{Föreningsform}
Föreningen är ansluten till Chalmers Studentkår och förbinder sig till att följa Chalmers studentkårs styrdokument samt att minst hälften av medlemmarna är chalmerister
