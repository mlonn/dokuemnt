\section{Årsmöte}
\subsection{Sammanträden} 
Styrelsen skall en gång per år kalla till årsmöte. Dag och tid för sammanträde fastställes av styrelsen i samband med kallelsen enligt §4.1. Mötet skall utlysas till alla medlemmar senast sju dagar innan mötet äger rum.
\subsection{Protokoll}

Vid årsmötet skall protokoll föras. Protokoll som förs skall innehålla anteckningar om ärendenas art, samtliga ställda och ej återtagna yrkanden, beslut samt särskilda yttranden och reservationer.

Protokoll justeras av årsmötets ordförande jämte två av årsmötet särskilt utsedda justeringspersoner.

Protokollet skall färdigställas och justeras inom trettio dagar efter årsmötet.

\subsection{Innehåll}
Vid ordinarie årsmöte skall följande struktur hållas:
\begin{enumerate}[label=§\arabic*]
    \item Mötets öppnande
\item Val av ordförande
\item Val av sekreterare
\item Val av mötets justerare tillika rösträknare
\item Närvarande
\item Mötets behöriga utlysande
\item Fastställande av mötets dagordning
\item Adjungeringar
\item Föregående mötesprotokoll
\item Meddelanden
\item Styrelsens verksamhetsberättelse
\item Styrelsens ekonomiska berättelse
\item Interpellationer
\item Propositioner
\item Motioner
\item Revisionsberättelse
\item Ansvarsfrihet
\item Personval
\begin{enumerate}[label*=.\arabic*]
    \item Val av ordförande till nästkommande styrelse
    \item Val av vice ordförande till nästkommande styrelse
    \item Val av kassör till nästkommande styrelse
    \item Val av ledamöter till nästkommande styrelse
    \item Val av två lekmannarevisorer
\end{enumerate}
\item Fastställande av medlemsavgift
\item Övriga frågor
\item Mötets avslutande
\end{enumerate}

Slutgiltig föredragningslista och till årsmötet hörande handlingar skall utständas till föreningens samtliga medlemmar senast tre dagar före årsmötet.
\subsection{Röstning}
Rösträtt tillkommer föreningens samtliga medlemmar inklusive styrelse, och föreningens revisorer.

Vid personval är den som erhållit flest röster vald. Vid lika antal röster äger ordförande rätten att lägga en extra röst på en av dessa.

Vid sakliga frågor är den mening som erhållit mer än hälften av antalet angivna rösterna mötets beslut. Om sådan röstövervikt ej nås, skall omröstning ske mellan de två som erhållit de flesta rösterna. Vid lika antal röster gäller som årsmötets beslut den mening som biträdes av mötesordförande.

Röstning med fullmakt får ej ske.

\subsection{Överklagande}
Beslut av årsmötet som strider mot Chalmers Studentkårs styrdokument får undanröjas av Chalmers Studentkårs styrelse. Sådant beslut skall tas upp till prövning om det begärs av en kårmedlem då det rör Chalmers Studentkårs stadga, eller av föreningens medlem då det rör föreningens stadga.