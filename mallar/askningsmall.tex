\documentclass[11pt, includeaddress]{classes/olbruket}
\usepackage{titlesec}
\usepackage{verbatimbox}
\usepackage{tabularx}
\usepackage{hyperref}

\usepackage{pgfkeys}

% Set up the keys.  Only the ones directly under /mytextfield
% can be accepted as options to the \mytextfield macro.
\pgfkeys{
 /mytextfield/.is family, /mytextfield,
 % Here are the options that a user can pass
 default/.style =
  {borderwidth = 0, dotwidth = 2cm, name=herp},
 borderwidth/.estore in = \mytextfieldBorderwidth,
 dotwidth/.estore in = \mytextfieldDotwidth,
 name/.estore in = \mytextfieldName,
}

\newdimen\longline
\longline=\textwidth\advance\longline-6cm
\def\LayoutTextField#1#2{#2} % override default in hyperref

\def\lbl#1{\hbox to \mytextfieldDotwidth{#1\dotfill\strut}}%
\def\labelline#1#2{\lbl{#1}\vbox{\hbox{\TextField[name=\mytextfieldName,width=#2, borderwidth=\mytextfieldBorderwidth]{\null}}\kern2pt\hrule}}

% We process the options first, then pass them to `\parbox` in the form of macros.
\newcommand\mytextfield[2][]{%
 \pgfkeys{/mytextfield, default, #1}%
 \hbox to \hsize{\labelline{#2}{\longline}}\vskip1.4ex
}

\titleformat{\paragraph}[hang]{\normalfont\normalsize\bfseries}{\theparagraph}{1em}{}
\titlespacing*{\paragraph}{0pt}{3.25ex plus 1ex minus 0.2ex}{0.7em}

\graphicspath{ {images/} }

\begin{document}

  \begin{Form}

\title{Äskningsansökan}

\makeheadfoot

\section*{Sökande}
För- och efternamn på den sökande eller organ inom Ölbruket. \\[14pt]
\mytextfield[name=namn1]{Namn:}

\section*{Ansvarig för äskningen}

\mytextfield[name=namn2]{Namn:}
\mytextfield[name=tel]{Telefon:}
\mytextfield[name=mail]{Mail:}

\section*{Motivering}
\TextField[multiline,width=\textwidth,height=11cm]{\mbox{ }}

\newpage

\setlength{\extrarowheight}{12pt}
\section*{Budget}
\begin{tabular}{ | p{12cm} | p{3cm} |}
	\multicolumn{1}{l}{\large{\textbf{Artikel}}} & \multicolumn{1}{l}{\large{\textbf{Utgift}}} \\
	\hline
	{\TextField[name=art1, borderwidth=0,width=12cm]{ }} & {\TextField[name=bug1, borderwidth=0,width=3cm]{ }} \\
	\hline
	{\TextField[name=art2, borderwidth=0,width=12cm]{ }} & {\TextField[name=bug2, borderwidth=0,width=3cm]{ }} \\
	\hline
	{\TextField[name=art3, borderwidth=0,width=12cm]{ }} & {\TextField[name=bug3, borderwidth=0,width=3cm]{ }} \\
	\hline
	{\TextField[name=art4, borderwidth=0,width=12cm]{ }} & {\TextField[name=bug4, borderwidth=0,width=3cm]{ }} \\
	\hline
	{\TextField[name=art5, borderwidth=0,width=12cm]{ }} & {\TextField[name=bug5, borderwidth=0,width=3cm]{ }} \\
	\hline
	{\TextField[name=art6, borderwidth=0,width=12cm]{ }} & {\TextField[name=bug6, borderwidth=0,width=3cm]{ }} \\
	\hline
	{\TextField[name=art7, borderwidth=0,width=12cm]{ }} & {\TextField[name=bug7, borderwidth=0,width=3cm]{ }} \\
	\hline
	{\TextField[name=art8, borderwidth=0,width=12cm]{ }} & {\TextField[name=bug8, borderwidth=0,width=3cm]{ }} \\
	\hline
	{\TextField[name=art9, borderwidth=0,width=12cm]{ }} & {\TextField[name=bug9, borderwidth=0,width=3cm]{ }} \\
	\hline
	\multicolumn{1}{r|}{ \large{\textbf{Total:}} } & \TextField[name=total, borderwidth=0,width=3cm]{ }\\ \cline{2-2}


\end{tabular}

\vspace{2cm}

\mytextfield[name=ort, dotwidth=5cm]{Ort och datum:}
\mytextfield[name=sig, dotwidth=5cm]{Signatur styrelsen:}
\mytextfield[name=fort, dotwidth=5cm]{Namnförtydligande:}

 \end{Form}
\end{document}
